\documentclass{article}
\usepackage[margin=1in]{geometry}
\usepackage{amsmath}
\usepackage{graphicx}
\usepackage{siunitx}
\usepackage{listings}
\usepackage{xcolor}
\usepackage{booktabs}
\usepackage{hyperref}
\definecolor{mygreen}{rgb}{0,0.6,0}
\definecolor{mygray}{rgb}{0.5,0.5,0.5}
\definecolor{mymauve}{rgb}{0.58,0,0.82}

\lstset{ %
  backgroundcolor=\color{white},   % choose the background color; you must add \usepackage{color} or \usepackage{xcolor}
  basicstyle=\scriptsize\ttfamily,    % the size of the fonts that are used for the code
  breakatwhitespace=false,         % sets if automatic breaks should only happen at whitespace
  breaklines=true,                 % sets automatic line breaking
  captionpos=b,                    % sets the caption-position to bottom
  commentstyle=\color{mygreen},    % comment style
  deletekeywords={},            % if you want to delete keywords from the given language
  escapeinside={\%*}{*)},          % if you want to add LaTeX within your code
  extendedchars=true,              % lets you use non-ASCII characters; for 8-bits encodings only, does not work with UTF-8
  frame=shadowbox,                    % adds a frame around the code
%  framexleftmarign=5mm,
  xleftmargin=10pt,
  xrightmargin=10pt,
  rulesepcolor=\color{gray},
  keywordstyle=\color{blue},       % keyword style
  language=Octave,                 % the language of the code
  morekeywords={*,...,fit,predint,export\_fig},            % if you want to add more keywords to the set
%  numbers=left,                    % where to put the line-numbers; possible values are (none, left, right)
  numbers=none,
  numbersep=5pt,                   % how far the line-numbers are from the code
  numberstyle=\tiny\color{mygray}, % the style that is used for the line-numbers
  rulecolor=\color{black},         % if not set, the frame-color may be changed on line-breaks within not-black text (e.g. comments (green here))
  showspaces=false,                % show spaces everywhere adding particular underscores; it overrides 'showstringspaces'
  showstringspaces=false,          % underline spaces within strings only
  showtabs=false,                  % show tabs within strings adding particular underscores
  stepnumber=1,                    % the step between two line-numbers. If it's 1, each line will be numbered
  stringstyle=\color{mymauve},     % string literal style
  tabsize=4,                       % sets default tabsize to 4 spaces
  caption=\lstname                   % show the filename of files included with \lstinputlisting; also try caption instead of title
}


\title{1.723 Final}
\author{Sachith  Dunatunga}

\begin{document}
\newcommand{\deriv}[2]{\frac{\partial #1}{ \partial #2}}
\newcommand{\nderiv}[3]{\frac{\partial^{#3} #1}{ \partial #2^{#3}}}
\newcommand{\dx}[1]{\deriv{#1}{x}}
\newcommand{\taylorexpf}[3]{#1_{#2} + \left(#3 \right) \dx{#1}\biggr\rvert_{#2} + \frac{1}{2}\left(#3 \right)^2 \nderiv{#1}{x}{2}\biggr\rvert_{#2} + \frac{1}{6}\left(#3 \right)^3\nderiv{#1}{x}{3}\biggr\rvert_{#2} + \frac{1}{24}\left(#3 \right)^4\nderiv{#1}{x}{4}\biggr\rvert_{#2} + O(h^5)}
\maketitle

\section{Problem 1}
\subsection{Part 1}
Since we know $\nabla^2 \Psi = - \omega$, we can write
\begin{align}
    \nabla^2 (\Psi_0 + \tilde{\Psi}) &= -\omega \\
    \nabla^2 \Psi_0 + \nabla^2 \tilde{\Psi} &= -\omega.
\end{align}
However, we know that for this problem $\Psi_0 = y$, since $\mathbf{u}_0 = {1, 0}$.
Therefore, the relationship is given by
\begin{align}
    \nabla^2 y + \nabla^2 \tilde{\Psi} &= -\omega \\
    \nabla^2 \tilde{\Psi} &= -\omega.
\end{align}

In Fourier space, derivatives are taken by multiplying by the wavenumber times the imaginary unit.
Thus, if the fourier transform of $\tilde{\Psi}$ is written as $\hat{\tilde{\Psi}}$, the fourier transform of $\nabla^2 \tilde{\Psi}$ is given by
\begin{align}
    \nabla^2 \tilde{\Psi} &= \nderiv{\tilde{\Psi}}{x}{2} + \nderiv{\tilde{\Psi}}{y}{2} \\
\implies \widehat{\nabla^2 \tilde{\Psi}} &= (ik_x)^2 \hat{\tilde{\Psi}} + (ik_y)^2 \hat{\tilde{\Psi}}\\
    \widehat{\nabla^2 \tilde{\Psi}} &= -(k^2_x + k^2_y) \hat{\tilde{\Psi}}.
\end{align}

Since at zero frequency in both x and y we have $exp(0) = 1$, the fourier transform turns into the average value of the signal.
In this case, since we only care about derivatives of the stream function $\Psi$, this constant is `free' and will not affect the results.

\subsection{Part 2}
From the previous problem we have
\begin{align}
    \widehat{\nabla^2 \tilde{\Psi}} &= -(k^2_x + k^2_y) \hat{\tilde{\Psi}}.
\end{align}
We can write the fourier transform of the equation
\begin{align}
    \nabla^2 \tilde{\Psi} &= -\omega
\end{align}
as
\begin{align}
    -(k^2_x + k^2_y) \hat{\tilde{\Psi}} &= -\hat{\omega} \\
\implies \hat{\tilde{\Psi}} &= \frac{\hat{\omega}}{k^2_x + k^2_y}.
\end{align}
Although we need to divide by $(k^2_x + k^2_y) = 0$, in this case since it only affects the mean value of $\Psi$ we can arbitrarily set this term to a nonzero constant.
This is because the mean value of the stream function is not important, only its derivatives, which are still accurate.

\clearpage
\appendix
\section{Code}
% \lstinputlisting[label=code:test]{../code/test.m}

\end{document}
