\documentclass{article}
\usepackage[margin=1in]{geometry}
\usepackage{amsmath}
\usepackage{graphicx}
\usepackage{listings}
\usepackage{xcolor}
\usepackage{hyperref}
\definecolor{mygreen}{rgb}{0,0.6,0}
\definecolor{mygray}{rgb}{0.5,0.5,0.5}
\definecolor{mymauve}{rgb}{0.58,0,0.82}

\lstset{ %
  backgroundcolor=\color{white},   % choose the background color; you must add \usepackage{color} or \usepackage{xcolor}
  basicstyle=\scriptsize\ttfamily,    % the size of the fonts that are used for the code
  breakatwhitespace=false,         % sets if automatic breaks should only happen at whitespace
  breaklines=true,                 % sets automatic line breaking
  captionpos=b,                    % sets the caption-position to bottom
  commentstyle=\color{mygreen},    % comment style
  deletekeywords={},            % if you want to delete keywords from the given language
  escapeinside={\%*}{*)},          % if you want to add LaTeX within your code
  extendedchars=true,              % lets you use non-ASCII characters; for 8-bits encodings only, does not work with UTF-8
  frame=shadowbox,                    % adds a frame around the code
%  framexleftmarign=5mm,
  xleftmargin=10pt,
  xrightmargin=10pt,
  rulesepcolor=\color{gray},
  keywordstyle=\color{blue},       % keyword style
  language=Octave,                 % the language of the code
  morekeywords={*,...,fit,predint,export\_fig},            % if you want to add more keywords to the set
%  numbers=left,                    % where to put the line-numbers; possible values are (none, left, right)
  numbers=none,
  numbersep=5pt,                   % how far the line-numbers are from the code
  numberstyle=\tiny\color{mygray}, % the style that is used for the line-numbers
  rulecolor=\color{black},         % if not set, the frame-color may be changed on line-breaks within not-black text (e.g. comments (green here))
  showspaces=false,                % show spaces everywhere adding particular underscores; it overrides 'showstringspaces'
  showstringspaces=false,          % underline spaces within strings only
  showtabs=false,                  % show tabs within strings adding particular underscores
  stepnumber=1,                    % the step between two line-numbers. If it's 1, each line will be numbered
  stringstyle=\color{mymauve},     % string literal style
  tabsize=4,                       % sets default tabsize to 4 spaces
  caption=\lstname                   % show the filename of files included with \lstinputlisting; also try caption instead of title
}


\title{1.723 HW1}
\author{Sachith  Dunatunga}

\begin{document}
\maketitle

%
%
%
\section{Problem 1}
\subsection{Part 1}
Plots of the two cross sections (black areas are solid, white areas are void), generated by the program in listing \ref{code:plotting}.

\begin{tabular}{c c}
\includegraphics{bot.pdf} & \includegraphics{top.pdf} \\
\end{tabular}

\subsection{Part 2}
I calculated the porosity $\phi$ by 
\begin{align}
    \phi &= \frac{\#\text{ solid pixels}}{\#\text{ total pixels}}. \\
\end{align}
The program in listing \ref{code:porosity} gave the values $\phi_{\mathrm{bottom}} = 0.8416$ and $\phi_{\mathrm{top}} = 0.7987$.

\subsection{Part 3}
Square windows of various sizes (centered at the origin) are used by the program in listing \ref{code:windows} to generate the following graph:
\begin{figure}[!h]
\centering
\includegraphics{windowed_porosity.pdf}A
\caption{The porosity is plotted as a function of the window edge size (length of the side of a square centered around the origin). Note the range of the y axis is between 0.7 and 1.0, so fluctuations in porosity beyond ~150-200 pixels per side are only a handful of percent. The distance from the edge to the center is half of this distance.}
\label{fig:windows}
\end{figure}

\subsection{Part 4}
Using figure \ref{fig:windows}, it looks like we at least need a window size of around 150-200 pixels per edge to get within a few percent of the overall porosity. This comes out to a total of around 22500 to 40000 pixels.
%
%
%
\section{Problem 2}
\subsection{Part 1}
Given that 1 darcy is $9.869233 \times 10^{-13} \mathrm{\ m}^2$, we have
\begin{align}
    1 \mathrm{\ d} &= 9.869233 \times 10^{-13} \mathrm{\ m}^2 \frac{10^2 \mathrm{\ cm}}{1 \mathrm{\ m}} \frac{10^2 \mathrm{\ cm}}{1 \mathrm{\ m}} \\
    1 \mathrm{\ d} &= 9.869233 \times 10^{-9} \mathrm{\ cm}^2
\end{align}

\subsection{Part 2}
Since kinematic viscosity $\nu = \mu / \rho$ and hydraulic conductivity $K = \rho g k / \mu$ from Nutting (1930), we can rearrange to get
\begin{align}
    K = kg/\nu.
\end{align}
Substituting, we get 
\begin{align}
    K_{\mathrm{w}} &= (9.869233 \times 10^{-9} \mathrm{\ cm}^2)(9.81 \times 10^2 \mathrm{\ cm} \mathrm{\ s}^{-2}) / (1.3 \times 10^{-2} \mathrm{\ cm}^2 \mathrm{\ s}^{-1}) \\
    &= 7.4475 \times 10^{-4} \mathrm{\ cm} \mathrm{\ s}^{-1}.
\end{align}
Similarly
\begin{align}
    K_{\mathrm{o}} &= (9.869233 \times 10^{-9} \mathrm{\ cm}^2)(9.81 \times 10^2 \mathrm{\ cm} \mathrm{\ s}^{-2}) / (1.8 \mathrm{\ cm}^2 \mathrm{\ s}^{-1}) \\
    &= 5.3787 \times 10^{-6} \mathrm{\ cm} \mathrm{\ s}^{-1}
\end{align}
%
%
%
\section{Problem 3}
For each set of data, we have the vector of darcy heights $\mathbf{\Delta h}$ and the vector of flow rates $\mathbf{q}$.
In each set, the cross sectional area ($A$) is constant and the height of the column ($L$) is constant.
We want the equation of best fit for $q(\Delta h) = \alpha \Delta h$, so we write
\begin{align}
    \mathbf{q} = \mathbf{\Delta h} \alpha.
\end{align}
We solve this using the normal equations, given by
\begin{align}
    \mathbf{\Delta h}^T \mathbf{q} = \mathbf{\Delta h}^T \mathbf{\Delta h} \alpha
\end{align}
which is rearranged to
\begin{align}
   \alpha = \frac{\mathbf{\Delta h}^T \mathbf{q}}{\mathbf{\Delta h}^T \mathbf{\Delta h}}.
\end{align}
Noting that $\alpha = KA/L$, we can rearrange for the conductivity K.
Then, using the kinematic viscosity of water from Problem 2 ($\nu_{w} = 0.013\mathrm{\ cm}^2 \mathrm{\ s}^{-1}$), we can obtain the permeabilities via $k = K\nu/g$.

This was done in the excel sheet, along with unit conversions, yielding the results
\begin{table}[!h]
\centering
\begin{tabular}{c | c c c}
Experiment & K [cm/s] & K [m/d] & k [md] \\
\hline
Atmospheric - Series 1 & 0.02849 & 24.617 & 38257 \\
Atmospheric - Series 2 & 0.01656 & 14.306 & 22232 \\
Atmospheric - Series 3 & 0.02153 & 18.599 & 28905 \\
Atmospheric - Series 4 & 0.02152 & 18.595 & 28899 \\
Pressurized & 0.02751 & 23.765 & 36933
\end{tabular}
\end{table}

Unfortunately, these results seem absurdly high (30 darcy is quite large for permeability, at most I would expect 1 darcy), and likewise for the hydraulic conductivity results.
However, I am unable to spot the error in units.

\appendix
\section{Programs}
\lstinputlisting[label=code:plotting]{sandstone_figures.py}
\lstinputlisting[label=code:porosity]{calculate_porosity.py}
\lstinputlisting[label=code:windows]{windowed_porosity.py}

\end{document}
