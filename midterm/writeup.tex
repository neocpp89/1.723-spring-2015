\documentclass{article}
\usepackage[margin=1in]{geometry}
\usepackage{amsmath}
\usepackage{graphicx}
\usepackage{siunitx}
\usepackage{listings}
\usepackage{xcolor}
\usepackage{hyperref}
\definecolor{mygreen}{rgb}{0,0.6,0}
\definecolor{mygray}{rgb}{0.5,0.5,0.5}
\definecolor{mymauve}{rgb}{0.58,0,0.82}

\lstset{ %
  backgroundcolor=\color{white},   % choose the background color; you must add \usepackage{color} or \usepackage{xcolor}
  basicstyle=\scriptsize\ttfamily,    % the size of the fonts that are used for the code
  breakatwhitespace=false,         % sets if automatic breaks should only happen at whitespace
  breaklines=true,                 % sets automatic line breaking
  captionpos=b,                    % sets the caption-position to bottom
  commentstyle=\color{mygreen},    % comment style
  deletekeywords={},            % if you want to delete keywords from the given language
  escapeinside={\%*}{*)},          % if you want to add LaTeX within your code
  extendedchars=true,              % lets you use non-ASCII characters; for 8-bits encodings only, does not work with UTF-8
  frame=shadowbox,                    % adds a frame around the code
%  framexleftmarign=5mm,
  xleftmargin=10pt,
  xrightmargin=10pt,
  rulesepcolor=\color{gray},
  keywordstyle=\color{blue},       % keyword style
  language=Octave,                 % the language of the code
  morekeywords={*,...,fit,predint,export\_fig},            % if you want to add more keywords to the set
%  numbers=left,                    % where to put the line-numbers; possible values are (none, left, right)
  numbers=none,
  numbersep=5pt,                   % how far the line-numbers are from the code
  numberstyle=\tiny\color{mygray}, % the style that is used for the line-numbers
  rulecolor=\color{black},         % if not set, the frame-color may be changed on line-breaks within not-black text (e.g. comments (green here))
  showspaces=false,                % show spaces everywhere adding particular underscores; it overrides 'showstringspaces'
  showstringspaces=false,          % underline spaces within strings only
  showtabs=false,                  % show tabs within strings adding particular underscores
  stepnumber=1,                    % the step between two line-numbers. If it's 1, each line will be numbered
  stringstyle=\color{mymauve},     % string literal style
  tabsize=4,                       % sets default tabsize to 4 spaces
  caption=\lstname                   % show the filename of files included with \lstinputlisting; also try caption instead of title
}


\title{1.723 Midterm}
\author{Sachith  Dunatunga}

\begin{document}
\maketitle

\section{Problem 1}
For the boundary conditions, we will start with the equation
\begin{align}
 \nabla \cdot \mathbf{u} = 0
\end{align}
and derive the finite volume method from there.
Taking an integral and using the divergence theorem, we can write
\begin{align}
\int_\Omega \nabla \cdot \mathbf{u}\ \mathrm{dV} &= 0 \\
\int_{\Gamma_{L}} \mathbf{u} \cdot \mathbf{n}\ \mathrm{dS} +
\int_{\Gamma_{D}} \mathbf{u} \cdot \mathbf{n}\ \mathrm{dS} +
\int_{\Gamma_{R}} \mathbf{u} \cdot \mathbf{n}\ \mathrm{dS} +
\int_{\Gamma_{U}} \mathbf{u} \cdot \mathbf{n}\ \mathrm{dS} &= 0.
\end{align}
For each of the boundaries, we simply set the corresponding integrated flux to zero.
We will derive the equation for the bottom boundary; the others are similar.

Since the flux through the bottom boundary is zero, the equation above is reduced to
\begin{align}
\int_{\Gamma_{L}} \mathbf{u} \cdot \mathbf{n}\ \mathrm{dS} +
\int_{\Gamma_{R}} \mathbf{u} \cdot \mathbf{n}\ \mathrm{dS} +
\int_{\Gamma_{U}} \mathbf{u} \cdot \mathbf{n}\ \mathrm{dS} &= 0.
\end{align}

The nondimensionalized darcy velocity is given by
\begin{align}
\mathbf{u} = - (\nabla p - c \nabla z).
\end{align}

Using the two point flux approximation, we can write the boundary integrals as
\begin{align}
\int_{\Gamma_{L}} \mathbf{u} \cdot \mathbf{n}\ \mathrm{dS} &= - \frac{\delta z}{\delta x} (p_{i,j-1} - p_{i,j}) \\
\int_{\Gamma_{R}} \mathbf{u} \cdot \mathbf{n}\ \mathrm{dS} &= \frac{\delta z}{\delta x} (p_{i,j} - p_{i,j+1}) \\
\int_{\Gamma_{U}} \mathbf{u} \cdot \mathbf{n}\ \mathrm{dS} &= - \frac{\delta x}{\delta z} \left( (p_{i-1,j} - p_{i,j}) - \delta z c_{i-1/2,j} \right).
\end{align}

\section{Problem 2}

% \appendix
% \section{Programs}
% \lstinputlisting[label=code:plotting]{sandstone_figures.py}

\end{document}
