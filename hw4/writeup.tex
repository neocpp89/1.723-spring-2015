\documentclass{article}
\usepackage[margin=1in]{geometry}
\usepackage{amsmath}
\usepackage{graphicx}
\usepackage{siunitx}
\usepackage{listings}
\usepackage{xcolor}
\usepackage{hyperref}
\definecolor{mygreen}{rgb}{0,0.6,0}
\definecolor{mygray}{rgb}{0.5,0.5,0.5}
\definecolor{mymauve}{rgb}{0.58,0,0.82}

\lstset{ %
  backgroundcolor=\color{white},   % choose the background color; you must add \usepackage{color} or \usepackage{xcolor}
  basicstyle=\scriptsize\ttfamily,    % the size of the fonts that are used for the code
  breakatwhitespace=false,         % sets if automatic breaks should only happen at whitespace
  breaklines=true,                 % sets automatic line breaking
  captionpos=b,                    % sets the caption-position to bottom
  commentstyle=\color{mygreen},    % comment style
  deletekeywords={},            % if you want to delete keywords from the given language
  escapeinside={\%*}{*)},          % if you want to add LaTeX within your code
  extendedchars=true,              % lets you use non-ASCII characters; for 8-bits encodings only, does not work with UTF-8
  frame=shadowbox,                    % adds a frame around the code
%  framexleftmarign=5mm,
  xleftmargin=10pt,
  xrightmargin=10pt,
  rulesepcolor=\color{gray},
  keywordstyle=\color{blue},       % keyword style
  language=Octave,                 % the language of the code
  morekeywords={*,...,fit,predint,export\_fig},            % if you want to add more keywords to the set
%  numbers=left,                    % where to put the line-numbers; possible values are (none, left, right)
  numbers=none,
  numbersep=5pt,                   % how far the line-numbers are from the code
  numberstyle=\tiny\color{mygray}, % the style that is used for the line-numbers
  rulecolor=\color{black},         % if not set, the frame-color may be changed on line-breaks within not-black text (e.g. comments (green here))
  showspaces=false,                % show spaces everywhere adding particular underscores; it overrides 'showstringspaces'
  showstringspaces=false,          % underline spaces within strings only
  showtabs=false,                  % show tabs within strings adding particular underscores
  stepnumber=1,                    % the step between two line-numbers. If it's 1, each line will be numbered
  stringstyle=\color{mymauve},     % string literal style
  tabsize=4,                       % sets default tabsize to 4 spaces
  caption=\lstname                   % show the filename of files included with \lstinputlisting; also try caption instead of title
}


\title{1.723 HW4}
\author{Sachith  Dunatunga}

\begin{document}
\maketitle

\section{Problem 1}
From the previous homework, we have the (nondimensionalized) 1D pressure equation given by
\begin{align}
    \frac{\partial p}{\partial t} + \frac{\partial}{\partial x} \left( -\lambda \frac{\partial p}{\partial x} \right) = 0.
\end{align}
Setting $\lambda = 1$ for ease of analysis, the two point flux approximation and Crank-Nicolson time stepping yields the discretization
\begin{align}
   p_n^{t+1} = p_n^{t} + \frac{\Delta t}{2 h^2}\left( (p_{n+1}^{t+1} - 2p_n^{t+1} + p_{n-1}^{t+1}) +  (p_{n+1}^{t} - 2p_n^{t} + p_{n-1}^{t}) \right).
    \label{eqn:cn-disc}
\end{align}

For spatial discretization error, we will drop the time superscripts. Taylor expanding the pressure around the point $x_n$ at time $t$ and evaulating at $x_{n \pm 1}$ generates the pair
\begin{align}
    p_{n+1} &= p_n + (x_{n+1} - x_n) \frac{\partial p}{\partial x} + \frac{1}{2} (x_{n+1} - x_n)^2 \frac{\partial^2 p}{\partial x^2} + \frac{1}{6} (x_{n+1} - x_n)^3 \frac{\partial^3 p}{\partial x^3} + O((x_{n+1} - x_n)^4) \\
    p_{n-1} &= p_n + (x_{n-1} - x_n) \frac{\partial p}{\partial x} + \frac{1}{2} (x_{n-1} - x_n)^2 \frac{\partial^2 p}{\partial x^2} + \frac{1}{6} (x_{n-1} - x_n)^3 \frac{\partial^3 p}{\partial x^3} + O((x_{n-1} - x_n)^4).
\end{align}
With a uniform grid spacing, these reduce down to
\begin{align}
    p_{n+1} &= p_n + h \frac{\partial p}{\partial x} + \frac{1}{2} h^2 \frac{\partial^2 p}{\partial x^2} + \frac{1}{6} h^3 \frac{\partial^3 p}{\partial x^3} + O(h^4) \\
    p_{n-1} &= p_n - h \frac{\partial p}{\partial x} + \frac{1}{2} h^2 \frac{\partial^2 p}{\partial x^2} - \frac{1}{6} h^3 \frac{\partial^3 p}{\partial x^3} + O(h^4).
\end{align}
Adding the two equations together and rearraning, we obtain
\begin{align}
    \frac{p_{n+1} - 2 p_n + p_{n-1}}{h^2} = \frac{\partial^2 p}{\partial x^2} + O(h^2).
\end{align}
Since the steady-state equation (with $\lambda = 1$)is given by
\begin{align}
\frac{\partial^2 p}{\partial x^2} = 0,
\end{align}
it is clear that our spatial discretization error is $O(h^2)$ (second order).

For the temporal discretization error, we first taylor expand the intermeditate pressure $p^{t+1/2}_n$ from both time $t$ and time $t+1$ (remembering that $t+ 1 - t = \Delta t$), yielding
\begin{align}
    p^{t+1/2}_{n} &= p^{t}_{n} + \frac{\Delta t}{2} \dot{p}^{t}_{n} + O(\Delta t^2) \\
    p^{t+1/2}_{n} &= p^{t+1}_{n} - \frac{\Delta t}{2} \dot{p}^{t+1}_{n} + O(\Delta t^2).
\end{align}
Subtracting the first equation from the second and rearranging results in
\begin{align}
    p^{t+1}_{n} = p^{t}_{n} + \frac{\Delta t}{2} (\dot{p}^{t}_{n} + \dot{p}^{t+1}_{n}) + O(\Delta t^2).
\end{align}
Since we know $\dot{p} = \lambda \frac{\partial^ p}{\partial x^2}$, we can substitute the spatial discretization at times $t$ and $t+1$ into the above equation.
Comparing the resulting expression to equation \eqref{eqn:cn-disc} shows that they are identical save for the $O(\Delta t)^2$, meaning the temporal discreization is second order as well. The Crank-Nicolson discretization is second order in space and time with the pressure equation.

\section{Problem 2}
The discretization of the one dimensional pressure equation with Backward Euler time stepping (and setting $\lambda = 1$ for ease of analysis) is given by
\begin{align}
   p_n^{t+1} = p_n^{t} + \frac{\Delta t}{h^2}\left( p_{n+1}^{t+1} - 2p_n^{t+1} + p_{n-1}^{t+1} \right).
    \label{eqn:be-disc}
\end{align}
For Von Neumann stability analysis, we assume exponential solutions in space and time and plug into the pressure. For a single wavenumber, we write this as
\begin{align}
    p^{t}_{n} &= C \exp(t+ikn) \\
    p^{t}_{n-1} &= C \exp(t+ikn) \exp(-ikh) \\
    p^{t}_{n+1} &= C \exp(t+ikn) \exp(ikh)\\
    p^{t+1}_{n-1} &= C \exp(t+ikn) \exp(\Delta t) \exp(-ikh)\\
    p^{t+1}_{n} &= C \exp(t+ikn) \exp(\Delta t)\\
    p^{t+1}_{n+1} &= C \exp(t+ikn) \exp(\Delta t) \exp(ikh).
\end{align}
Substituting these equations into the discretization results in
\begin{align*}
    C \exp(t+ikn) \exp(\Delta t) &= C \exp(t+ikn) + \frac{\Delta t}{h^2} (C \exp(t+ikn) \exp(\Delta t) \exp(ikh) - \\
                                & 2 C \exp(t+ikn) \exp(\Delta t) + C \exp(t+ikn) \exp(\Delta t) \exp(-ikh)),
\end{align*}
which can be rearranged to
\begin{align}
    \exp(\Delta t) &= 1 + \exp(\Delta t) \frac{\Delta t}{h^2} \left( \exp(ikh) - 2 + \exp(-ikh) \right),
\end{align}

Since the amplification of errors is proportional to $p^{t+1}_{n} / p^{t}_{n}$, we have
\begin{align}
    G \equiv \exp(\Delta t) &\implies G = 1 + G \frac{\Delta t}{h^2} \left( \exp(ikh) - 2 + \exp(-ikh) \right) \\
    G &= \frac{h^2} {h^2 - \Delta t \left( \exp(ikh) - 2 + \exp(-ikh) \right)},
\end{align}
which can be rewritten as
\begin{align}
    G &= \frac{1} {1 + R \sin^2 (kh/2)}
\end{align}
where $R = \frac{4 \Delta t}{h^2}$. Since $\sin^2(x) \ge 0$, the amplification factor $G \le 1$, so the method is unconditionally stable.

\section{Problem 3}
We first write the linear advection equation as
\begin{align}
    \frac{\partial c}{\partial t} + u \frac{\partial c}{\partial x} = 0.
\end{align}
Without loss of generality, assume that $u$ is positive. The discretization with an upwind derivative and the Backward Euler method is then given by
\begin{align}
    \frac{c^{t+1}_n - c^t_n}{\Delta t} + u \frac{c^{t+1}_n - c^{t+1}_{n-1}}{h} = 0,
\end{align}
while for Forward Euler it is given by
\begin{align}
    \frac{c^{t+1}_n - c^t_n}{\Delta t} + u \frac{c^{t}_n - c^{t}_{n-1}}{h} = 0.
\end{align}

To analyze the discretization error, we Taylor expand $c$ at time $t$ and position $n$ around the time $t+1$ (for Backward Euler).
In Forward Euler, we instead expand $c$ at time $t+1$ and position $n$ around the time $t$.
We also Taylor expand $c$ at arbitrary time $t$ and position $n$ around position $n-1$. These three expansions are given by
\begin{align}
    c^{t}_{n} &= c^{t+1} - \Delta t \dot{c}^{t+1}_n + O(\Delta t^2) \\
    c^{t+1}_{n} &= c^{t}_n + \Delta t \dot{c}^{t}_n + O(\Delta t^2) \\
    c^{t}_{n} &= c^{t}_{n-1} + h \left( \frac{\partial c}{\partial x} \right)^{t}_{n-1} + O(h^2)
\end{align}
Substitution of the first and third equation (multiplied by $u$) into the backward euler discretization yields
\begin{align}
    \dot{c}^{t}_n + u - O(\Delta t) + \left( \frac{\partial c}{\partial x} \right)^{t}_{n-1} - O(h) = 0,
\end{align}
so the modified equation is then
\begin{align}
    \frac{\partial c}{\partial t} + u \frac{\partial c}{\partial x} = O(h) + O(\Delta t).
\end{align}
Thus, the method is first order in both space and time.

Repeating the analysis for the forward euler scheme (using the second and third equations this time), we arrive at the modified equation
\begin{align}
    \frac{\partial c}{\partial t} + u \frac{\partial c}{\partial x} = O(h) - O(\Delta t).
\end{align}

Note that the $O(h)$ term contains a diffusion-like expression ($\frac{\partial^2 c}{\partial x^2}$) due to the Taylor expansion.
We see that from the signs of the $O(h)$ and $O(\Delta t)$ the diffusion-like term is larger for the Backward Euler method than the Forward Euler, so we expect more diffusion from the Backward Euler scheme.

% \appendix
% \section{Programs}
% \lstinputlisting[label=code:plotting]{sandstone_figures.py}

\end{document}
