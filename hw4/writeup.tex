\documentclass{article}
\usepackage[margin=1in]{geometry}
\usepackage{amsmath}
\usepackage{graphicx}
\usepackage{siunitx}
\usepackage{listings}
\usepackage{xcolor}
\usepackage{hyperref}
\definecolor{mygreen}{rgb}{0,0.6,0}
\definecolor{mygray}{rgb}{0.5,0.5,0.5}
\definecolor{mymauve}{rgb}{0.58,0,0.82}

\lstset{ %
  backgroundcolor=\color{white},   % choose the background color; you must add \usepackage{color} or \usepackage{xcolor}
  basicstyle=\scriptsize\ttfamily,    % the size of the fonts that are used for the code
  breakatwhitespace=false,         % sets if automatic breaks should only happen at whitespace
  breaklines=true,                 % sets automatic line breaking
  captionpos=b,                    % sets the caption-position to bottom
  commentstyle=\color{mygreen},    % comment style
  deletekeywords={},            % if you want to delete keywords from the given language
  escapeinside={\%*}{*)},          % if you want to add LaTeX within your code
  extendedchars=true,              % lets you use non-ASCII characters; for 8-bits encodings only, does not work with UTF-8
  frame=shadowbox,                    % adds a frame around the code
%  framexleftmarign=5mm,
  xleftmargin=10pt,
  xrightmargin=10pt,
  rulesepcolor=\color{gray},
  keywordstyle=\color{blue},       % keyword style
  language=Octave,                 % the language of the code
  morekeywords={*,...,fit,predint,export\_fig},            % if you want to add more keywords to the set
%  numbers=left,                    % where to put the line-numbers; possible values are (none, left, right)
  numbers=none,
  numbersep=5pt,                   % how far the line-numbers are from the code
  numberstyle=\tiny\color{mygray}, % the style that is used for the line-numbers
  rulecolor=\color{black},         % if not set, the frame-color may be changed on line-breaks within not-black text (e.g. comments (green here))
  showspaces=false,                % show spaces everywhere adding particular underscores; it overrides 'showstringspaces'
  showstringspaces=false,          % underline spaces within strings only
  showtabs=false,                  % show tabs within strings adding particular underscores
  stepnumber=1,                    % the step between two line-numbers. If it's 1, each line will be numbered
  stringstyle=\color{mymauve},     % string literal style
  tabsize=4,                       % sets default tabsize to 4 spaces
  caption=\lstname                   % show the filename of files included with \lstinputlisting; also try caption instead of title
}


\title{1.723 HW4}
\author{Sachith  Dunatunga}

\begin{document}
\maketitle

\section{Problem 1}
From the previous homework, we have the (nondimensionalized) 1D pressure equation given by
\begin{align}
    \frac{\partial p}{\partial t} + \frac{\partial}{\partial x} \left( -\lambda \frac{\partial p}{\partial x} \right) = 0.
\end{align}
Setting $\lambda = 1$ for ease of analysis, the two point flux approximation and Crank-Nicolson time stepping yields the discretization
\begin{align}
   p_n^{t+1} = p_n^{t} + \frac{\Delta t}{2 h^2}\left( (p_{n+1}^{t+1} - 2p_n^{t+1} + p_{n-1}^{t+1}) +  (p_{n+1}^{t} - 2p_n^{t} + p_{n-1}^{t}) \right).
\end{align}
For spatial discretization error, we will drop the time superscripts. Taylor expanding the pressure around the point $x_n$ at time $t$ and evaulating at $x_{n \pm 1}$ generates the pair
\begin{align}
    p_{n+1} &= p_n + (x_{n+1} - x_n) \frac{\partial p}{\partial x} + \frac{1}{2} (x_{n+1} - x_n)^2 \frac{\partial^2 p}{\partial x^2} + \frac{1}{6} (x_{n+1} - x_n)^3 \frac{\partial^3 p}{\partial x^3} + O((x_{n+1} - x_n)^4) \\
    p_{n-1} &= p_n + (x_{n-1} - x_n) \frac{\partial p}{\partial x} + \frac{1}{2} (x_{n-1} - x_n)^2 \frac{\partial^2 p}{\partial x^2} + \frac{1}{6} (x_{n-1} - x_n)^3 \frac{\partial^3 p}{\partial x^3} + O((x_{n-1} - x_n)^4).
\end{align}
With a uniform grid spacing, these reduce down to
\begin{align}
    p_{n+1} &= p_n + h \frac{\partial p}{\partial x} + \frac{1}{2} h^2 \frac{\partial^2 p}{\partial x^2} + \frac{1}{6} h^3 \frac{\partial^3 p}{\partial x^3} + O(h^4) \\
    p_{n-1} &= p_n - h \frac{\partial p}{\partial x} + \frac{1}{2} h^2 \frac{\partial^2 p}{\partial x^2} - \frac{1}{6} h^3 \frac{\partial^3 p}{\partial x^3} + O(h^4).
\end{align}
Adding the two equations together and rearraning, we obtain
\begin{align}
    \frac{p_{n+1} - 2 p_n + p_{n-1}}{h^2} = \frac{\partial^2 p}{\partial x^2} + O(h^2).
\end{align}
Since the steady-state equation (with $\lambda = 1$)is given by
\begin{align}
\frac{\partial^2 p}{\partial x^2} = 0,
\end{align}
it is clear that our spatial discretization error is $O(h^2)$ (second order).

% \appendix
% \section{Programs}
% \lstinputlisting[label=code:plotting]{sandstone_figures.py}

\end{document}
