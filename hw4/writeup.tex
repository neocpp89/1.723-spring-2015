\documentclass{article}
\usepackage[margin=1in]{geometry}
\usepackage{amsmath}
\usepackage{graphicx}
\usepackage{siunitx}
\usepackage{listings}
\usepackage{xcolor}
\usepackage{hyperref}
\definecolor{mygreen}{rgb}{0,0.6,0}
\definecolor{mygray}{rgb}{0.5,0.5,0.5}
\definecolor{mymauve}{rgb}{0.58,0,0.82}

\lstset{ %
  backgroundcolor=\color{white},   % choose the background color; you must add \usepackage{color} or \usepackage{xcolor}
  basicstyle=\scriptsize\ttfamily,    % the size of the fonts that are used for the code
  breakatwhitespace=false,         % sets if automatic breaks should only happen at whitespace
  breaklines=true,                 % sets automatic line breaking
  captionpos=b,                    % sets the caption-position to bottom
  commentstyle=\color{mygreen},    % comment style
  deletekeywords={},            % if you want to delete keywords from the given language
  escapeinside={\%*}{*)},          % if you want to add LaTeX within your code
  extendedchars=true,              % lets you use non-ASCII characters; for 8-bits encodings only, does not work with UTF-8
  frame=shadowbox,                    % adds a frame around the code
%  framexleftmarign=5mm,
  xleftmargin=10pt,
  xrightmargin=10pt,
  rulesepcolor=\color{gray},
  keywordstyle=\color{blue},       % keyword style
  language=Octave,                 % the language of the code
  morekeywords={*,...,fit,predint,export\_fig},            % if you want to add more keywords to the set
%  numbers=left,                    % where to put the line-numbers; possible values are (none, left, right)
  numbers=none,
  numbersep=5pt,                   % how far the line-numbers are from the code
  numberstyle=\tiny\color{mygray}, % the style that is used for the line-numbers
  rulecolor=\color{black},         % if not set, the frame-color may be changed on line-breaks within not-black text (e.g. comments (green here))
  showspaces=false,                % show spaces everywhere adding particular underscores; it overrides 'showstringspaces'
  showstringspaces=false,          % underline spaces within strings only
  showtabs=false,                  % show tabs within strings adding particular underscores
  stepnumber=1,                    % the step between two line-numbers. If it's 1, each line will be numbered
  stringstyle=\color{mymauve},     % string literal style
  tabsize=4,                       % sets default tabsize to 4 spaces
  caption=\lstname                   % show the filename of files included with \lstinputlisting; also try caption instead of title
}


\title{1.723 HW4}
\author{Sachith  Dunatunga}

\begin{document}
\maketitle

\section{Problem 1}
From the previous homework, we have the (nondimensionalized) 1D pressure equation given by
\begin{align}
    \frac{\partial p}{\partial t} + \frac{\partial}{\partial x} \left( -\lambda \frac{\partial p}{\partial x} \right) = 0.
\end{align}
Setting $\lambda = 1$ for ease of analysis, the two point flux approximation and Crank-Nicolson time stepping yields the discretization
\begin{align}
   p_n^{t+1} = p_n^{t} + \frac{\Delta t}{2 h^2}\left( (p_{n+1}^{t+1} - 2p_n^{t+1} + p_{n-1}^{t+1}) +  (p_{n+1}^{t} - 2p_n^{t} + p_{n-1}^{t}) \right).
    \label{eqn:cn-disc}
\end{align}

For spatial discretization error, we will drop the time superscripts. Taylor expanding the pressure around the point $x_n$ at time $t$ and evaulating at $x_{n \pm 1}$ generates the pair
\begin{align}
    p_{n+1} &= p_n + (x_{n+1} - x_n) \frac{\partial p}{\partial x} + \frac{1}{2} (x_{n+1} - x_n)^2 \frac{\partial^2 p}{\partial x^2} + \frac{1}{6} (x_{n+1} - x_n)^3 \frac{\partial^3 p}{\partial x^3} + O((x_{n+1} - x_n)^4) \\
    p_{n-1} &= p_n + (x_{n-1} - x_n) \frac{\partial p}{\partial x} + \frac{1}{2} (x_{n-1} - x_n)^2 \frac{\partial^2 p}{\partial x^2} + \frac{1}{6} (x_{n-1} - x_n)^3 \frac{\partial^3 p}{\partial x^3} + O((x_{n-1} - x_n)^4).
\end{align}
With a uniform grid spacing, these reduce down to
\begin{align}
    p_{n+1} &= p_n + h \frac{\partial p}{\partial x} + \frac{1}{2} h^2 \frac{\partial^2 p}{\partial x^2} + \frac{1}{6} h^3 \frac{\partial^3 p}{\partial x^3} + O(h^4) \\
    p_{n-1} &= p_n - h \frac{\partial p}{\partial x} + \frac{1}{2} h^2 \frac{\partial^2 p}{\partial x^2} - \frac{1}{6} h^3 \frac{\partial^3 p}{\partial x^3} + O(h^4).
\end{align}
Adding the two equations together and rearraning, we obtain
\begin{align}
    \frac{p_{n+1} - 2 p_n + p_{n-1}}{h^2} = \frac{\partial^2 p}{\partial x^2} + O(h^2).
\end{align}
Since the steady-state equation (with $\lambda = 1$)is given by
\begin{align}
\frac{\partial^2 p}{\partial x^2} = 0,
\end{align}
it is clear that our spatial discretization error is $O(h^2)$ (second order).

For the temporal discretization error, we first taylor expand the intermeditate pressure $p^{t+1/2}_n$ from both time $t$ and time $t+1$ (remembering that $t+ 1 - t = \Delta t$), yielding
\begin{align}
    p^{t+1/2}_{n} &= p^{t}_{n} + \frac{\Delta t}{2} \dot{p}^{t}_{n} + O(\Delta t^2) \\
    p^{t+1/2}_{n} &= p^{t+1}_{n} - \frac{\Delta t}{2} \dot{p}^{t+1}_{n} + O(\Delta t^2).
\end{align}
Subtracting the first equation from the second and rearranging results in
\begin{align}
    p^{t+1}_{n} = p^{t}_{n} + \frac{\Delta t}{2} (\dot{p}^{t}_{n} + \dot{p}^{t+1}_{n}) + O(\Delta t^2).
\end{align}
Since we know $\dot{p} = \lambda \frac{\partial^ p}{\partial x^2}$, we can substitute the spatial discretization at times $t$ and $t+1$ into the above equation.
Comparing the resulting expression to equation \eqref{eqn:cn-disc} shows that they are identical save for the $O(\Delta t)^2$, meaning the temporal discreization is second order as well. The Crank-Nicolson discretization is second order in space and time with the pressure equation.

\section{Problem 2}
The discretization of the one dimensional pressure equation with Backward Euler time stepping (and setting $\lambda = 1$ for ease of analysis) is given by
\begin{align}
   p_n^{t+1} = p_n^{t} + \frac{\Delta t}{h^2}\left( p_{n+1}^{t+1} - 2p_n^{t+1} + p_{n-1}^{t+1} \right).
    \label{eqn:be-disc}
\end{align}
For Von Neumann stability analysis, we assume exponential solutions in space and time and plug into the pressure. For a single wavenumber, we write this as
\begin{align}
    p^{t}_{n} &= C \exp(\omega t+ikn) \\
    p^{t}_{n-1} &= C \exp(\omega t+ikn) \exp(-ikh) \\
    p^{t}_{n+1} &= C \exp(\omega t+ikn) \exp(ikh)\\
    p^{t+1}_{n-1} &= C \exp(\omega t+ikn) \exp(\Delta t) \exp(-ikh)\\
    p^{t+1}_{n} &= C \exp(\omega t+ikn) \exp(\Delta t)\\
    p^{t+1}_{n+1} &= C \exp(\omega t+ikn) \exp(\Delta t) \exp(ikh).
\end{align}
Substituting these equations into the discretization results in
\begin{align*}
    C \exp(\omega t+ikn) \exp(\Delta t) &= C \exp(\omega t+ikn) + \frac{\Delta t}{h^2} (C \exp(\omega t+ikn) \exp(\Delta t) \exp(ikh) - \\
                                & 2 C \exp(\omega t+ikn) \exp(\Delta t) + C \exp(\omega t+ikn) \exp(\Delta t) \exp(-ikh)),
\end{align*}
which can be rearranged to
\begin{align}
    \exp(\omega \Delta t) &= 1 + \exp(\omega \Delta t) \frac{\Delta t}{h^2} \left( \exp(ikh) - 2 + \exp(-ikh) \right),
\end{align}

Since the amplification of errors is proportional to $p^{t+1}_{n} / p^{t}_{n}$, we have
\begin{align}
    G \equiv \exp(\omega \Delta t) &\implies G = 1 + G \frac{\Delta t}{h^2} \left( \exp(ikh) - 2 + \exp(-ikh) \right) \\
    G &= \frac{h^2} {h^2 - \Delta t \left( \exp(ikh) - 2 + \exp(-ikh) \right)},
\end{align}
which can be rewritten as
\begin{align}
    G &= \frac{1} {1 + R \sin^2 (kh/2)}
\end{align}
where $R = \frac{4 \Delta t}{h^2}$. Since $\sin^2(x) \ge 0$, the amplification factor $G \le 1$, so the method is unconditionally stable.

\section{Problem 3}
We first write the linear advection equation as
\begin{align}
    \frac{\partial c}{\partial t} + u \frac{\partial c}{\partial x} = 0.
\end{align}
Without loss of generality, assume that $u$ is positive. The discretization with an upwind derivative and the Backward Euler method is then given by
\begin{align}
    \frac{c^{t+1}_n - c^t_n}{\Delta t} + u \frac{c^{t+1}_n - c^{t+1}_{n-1}}{h} = 0,
\end{align}
while for Forward Euler it is given by
\begin{align}
    \frac{c^{t+1}_n - c^t_n}{\Delta t} + u \frac{c^{t}_n - c^{t}_{n-1}}{h} = 0.
\end{align}

To analyze the discretization error, we Taylor expand $c$ at time $t$ and position $n$ around the time $t+1$ (for Backward Euler).
In Forward Euler, we instead expand $c$ at time $t+1$ and position $n$ around the time $t$.
We also Taylor expand $c$ at arbitrary time $t$ and position $n$ around position $n-1$. These three expansions are given by
\begin{align}
    c^{t}_{n} &= c^{t+1} - \Delta t \dot{c}^{t+1}_n + O(\Delta t^2) \\
    c^{t+1}_{n} &= c^{t}_n + \Delta t \dot{c}^{t}_n + O(\Delta t^2) \\
    c^{t}_{n} &= c^{t}_{n-1} + h \left( \frac{\partial c}{\partial x} \right)^{t}_{n-1} + O(h^2)
\end{align}
Substitution of the first and third equation (multiplied by $u$) into the backward euler discretization yields
\begin{align}
    \dot{c}^{t}_n + u - O(\Delta t) + \left( \frac{\partial c}{\partial x} \right)^{t}_{n-1} - O(h) = 0,
\end{align}
so the modified equation is then
\begin{align}
    \frac{\partial c}{\partial t} + u \frac{\partial c}{\partial x} = O(h) + O(\Delta t).
\end{align}
Thus, the method is first order in both space and time.

Repeating the analysis for the forward euler scheme (using the second and third equations this time), we arrive at the modified equation
\begin{align}
    \frac{\partial c}{\partial t} + u \frac{\partial c}{\partial x} = O(h) - O(\Delta t).
\end{align}

Note that the $O(h)$ term contains a diffusion-like expression ($\frac{\partial^2 c}{\partial x^2}$) due to the Taylor expansion.
We see that from the signs of the $O(h)$ and $O(\Delta t)$ the diffusion-like term is larger for the Backward Euler method than the Forward Euler, so we expect more diffusion from the Backward Euler scheme.

\section{Problem 4}
The discretization of the linear advection equation using a Forward Euler and centered two point flux approximation is given by
\begin{align}
    \frac{c^{t+1}_n - c^t_n}{\Delta t} + u \frac{c^{t}_{n+1} - c^{t}_{n-1}}{2h} = 0.
\end{align}
As before, we assume an exponential solution in both time and space to see how errors grow, leading to the defintions
\begin{align}
    c^{t}_{n} &= C \exp(\omega t+ikn) \\
    c^{t}_{n-1} &= C \exp(\omega t+ikn) \exp(-ikh) \\
    c^{t}_{n+1} &= C \exp(\omega t+ikn) \exp(ikh)\\
    c^{t+1}_{n} &= C \exp(\omega t+ikn) \exp(\Delta t).
\end{align}
Substitution into the discretization results in
\begin{align*}
C \exp(\omega t+ikn) \exp(\Delta t) &- C \exp(\omega t+ikn) + \\
    & \frac{u \Delta t}{2h} (C \exp(\omega t+ikn) \exp(ikh) - C \exp(\omega t+ikn) \exp(-ikh)) = 0,
\end{align*}
which becomes (after factoring)
\begin{align}
\exp(\omega \Delta t) - 1 + \frac{u \Delta t}{2h} (\exp(ikh) - \exp(-ikh)) = 0,
\end{align}
and can be rewritten as
\begin{align}
\exp(\omega \Delta t) - 1 + \frac{u \Delta t}{h} i \sin(kh) = 0,
\end{align}

Since the amplification factor $G = c^{t+1}_n / c^{t}_{n} \equiv \exp(\omega \Delta t)$, we have
\begin{align}
    G = 1 - \frac{u \Delta t}{h} i \sin(kh),
\end{align}
and the magnitude is given by
\begin{align}
    \| G \| = \sqrt{G \bar{G}} = \sqrt{1 + \left( \frac{u \Delta t}{h} \sin(kh) \right)^2}.
\end{align}
This is clearly greater than one (unless $u = 0$, in which case it is equal to one and the equation is uninteresting), so the method is unconditionally unstable.

\section{Problem 5}
\subsection{Part 1}
The equation we want to nondimensionalize is the linear tracer transport equation, given by
\begin{align}
    \phi \frac{\partial c}{\partial t} + \frac{\partial}{\partial x} \left(uc - D \frac{\partial c}{\partial x} \right) = 0.
\end{align}
As in the previous homework, we first note that we can use the chain rule in combination with the definitions to nondimensionalize the equation, which leads to
\begin{align}
    \phi_c \phi_D \frac{c_c \partial c_D}{T_c \partial t_D} + \frac{\partial}{L \partial x_D} \left(u_c u_D c_c c_D - D_c D_D \frac{c_c \partial c_D}{L \partial x_D} \right) = 0.
\end{align}
We can collect the coefficients to obtain
\begin{align}
    \phi_D \frac{\partial c_D}{\partial t_D} + \frac{\partial}{\partial x_D} \left(\frac{T_c u_c}{\phi_c L} u_D c_D - D_D \frac{D_c T_c}{\phi_c L^2} \frac{\partial c_D}{\partial x_D} \right) = 0.
\end{align}
Setting the coefficient left term inside the brackets to unity allows us to find the characteristic advective time
\begin{align}
    T_c = \frac{\phi_c L}{u_c}.
\end{align}

\subsection{Part 2}
Going back to the previous equations, we see that the Peclet number is given by
\begin{align}
    \mathrm{Pe} = \frac{\phi_c L^2}{T_c D_c},
\end{align}
which, using expression for the charcteristic advective time $T_c$, becomes
\begin{align}
    \mathrm{Pe} = \frac{u_c L}{D_c}.
\end{align}
Since $u_c$ is the advective velocity and $D_c / L$ looks like a velocity due to diffusion, the Peclet number is the ratio of advective and diffusive velocities.

\section{Problem 6}
\subsection{Part 1}
The plots of the anayltical solution given by
\begin{align}
    c(x,t) = \frac{1}{2} \mathrm{erfc} \left( \frac{x - t}{2 \sqrt{t / \mathrm{Pe}}} \right)
\end{align}
for the linear tracer transport equation are shown in figure \ref{fig:pe-analytical}.
\begin{figure}
\centering
\begin{tabular}{c c}
    \includegraphics[scale=1.0]{figs/analytic_1.pdf} &
    \includegraphics[scale=1.0]{figs/analytic_10.pdf} \\
    \includegraphics[scale=1.0]{figs/analytic_100.pdf} &
    \\
\end{tabular}
\caption{Analytical solution curves for the linear tracer transport equation at different times with varying Peclet numbers. We see that higher Peclet numbers result in a sharper interface. This makes sense with the interpretation given in the previous problem, as the advective velocity is higher than the diffusive velocity with a sharp front.}
\label{fig:pe-analytical}
\end{figure}

Due to the diffusion term, it is hard to say exactly when the tracer ``reaches'' the right boundary.
However, from the form of the analytical solution, it is clear that $c(1,1) = 0.5$ regardless of Peclet number.
With increasing Peclet number, this appears more and more like a shock, so at extremely high Peclet numbers the tracer concentration will go from 0 to 1 just before and just after t=1 respectively.
At lower Peclect numbers the tracer diffuses more, so the transition is not as dramatic.

\subsection{Part 2}
The equation we want to discretize is given by
\begin{align}
    \frac{\partial c}{\partial t} + \frac{\partial}{\partial x} \left(c - \frac{1}{\mathrm{Pe}} \frac{\partial c}{\partial x} \right) = 0.
\end{align}
Following the finite volume approach, we break the system up into cells and integrate assuming piecewise constant properties over each cell, yielding for cell $n$
\begin{align}
    \frac{\partial c}{\partial t} h  + \left( c - \frac{1}{\mathrm{Pe}} \frac{\partial c}{\partial x} \right)\biggr\rvert^{n+1/2}_{n-1/2}  = 0.
\end{align}
Since we are using the trapezoidal rule in time, we rearrange and discretize the temporal derivative as
\begin{align}
c^{t+1}_{n} - c^{t}_{n} = \frac{-\Delta t}{h}\left( \theta \left(c - \frac{1}{\mathrm{Pe}} \frac{\partial c}{\partial x} \right)^{t+1}\biggr\rvert^{n+1/2}_{n-1/2} + (1 - \theta) \left(c - \frac{1}{\mathrm{Pe}} \frac{\partial c}{\partial x} \right)^{t}\biggr\rvert^{n+1/2}_{n-1/2} \right).
\label{eqn:with-flux}
\end{align}
Using the two point flux approximation for the diffusive term and upwinding for the convective part, we arrive at
\begin{align}
c^{t+1}_{n} - c^{t}_{n} = \frac{-\Delta t}{h}\left( \theta \left(c^{t+1}_n - c^{t+1}_{n-1} - \frac{1}{\mathrm{Pe}} \frac{c^{t+1}_{n+1} - 2c^{t+1}_{n} + c^{t+1}_{n-1}}{h} \right) + (1 - \theta) \left(c^t_{n} - c^{t}_{n-1} - \frac{1}{\mathrm{Pe}} \frac{c^{t}_{n+1} - 2c^{t}_{n} + c^{t}_{n-1}}{h} \right) \right)
\end{align}
as the discretization for the interior points.

For the boundary conditions, we see that both of them are setting the flux into and out of the first and last cells.
We can substitute these fluxes directly into equation \eqref{eqn:with-flux}.
This yields
\begin{align}
c^{t+1}_{1} - c^{t}_{1} = \frac{-\Delta t}{h}\left( \theta \left(c^{t+1}_{1} - \frac{1}{\mathrm{Pe}} \frac{c^{t+1}_{2} - c^{t+1}_{1}}{h} - 1 \right) + (1 - \theta) \left( c^{t}_{1} - \frac{1}{\mathrm{Pe}} \frac{c^{t}_{2} - c^{t}_{1}}{h} - 1 \right) \right)
\end{align}
for the leftmost point (index 1)  and
\begin{align}
c^{t+1}_{N} - c^{t}_{N} = \frac{-\Delta t}{h}\left( \theta \left(c^{t+1}_{N} - c^{t+1}_{N-1} - \frac{1}{\mathrm{Pe}} \frac{c^{t+1}_{N-1} - c^{t+1}_{N}}{h} \right) + (1 - \theta) \left( c^{t}_{N} - c^{t}_{N-1} - \frac{1}{\mathrm{Pe}} \frac{c^{t}_{N-1} - c^{t}_{N}}{h} \right) \right)
\end{align}
for the rightmost point (index N).

% \appendix
% \section{Programs}
% \lstinputlisting[label=code:plotting]{sandstone_figures.py}

\end{document}
