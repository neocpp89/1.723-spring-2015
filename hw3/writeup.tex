\documentclass{article}
\usepackage[margin=1in]{geometry}
\usepackage{amsmath}
\usepackage{graphicx}
\usepackage{siunitx}
\usepackage{listings}
\usepackage{xcolor}
\usepackage{hyperref}
\definecolor{mygreen}{rgb}{0,0.6,0}
\definecolor{mygray}{rgb}{0.5,0.5,0.5}
\definecolor{mymauve}{rgb}{0.58,0,0.82}

\lstset{ %
  backgroundcolor=\color{white},   % choose the background color; you must add \usepackage{color} or \usepackage{xcolor}
  basicstyle=\scriptsize\ttfamily,    % the size of the fonts that are used for the code
  breakatwhitespace=false,         % sets if automatic breaks should only happen at whitespace
  breaklines=true,                 % sets automatic line breaking
  captionpos=b,                    % sets the caption-position to bottom
  commentstyle=\color{mygreen},    % comment style
  deletekeywords={},            % if you want to delete keywords from the given language
  escapeinside={\%*}{*)},          % if you want to add LaTeX within your code
  extendedchars=true,              % lets you use non-ASCII characters; for 8-bits encodings only, does not work with UTF-8
  frame=shadowbox,                    % adds a frame around the code
%  framexleftmarign=5mm,
  xleftmargin=10pt,
  xrightmargin=10pt,
  rulesepcolor=\color{gray},
  keywordstyle=\color{blue},       % keyword style
  language=Octave,                 % the language of the code
  morekeywords={*,...,fit,predint,export\_fig},            % if you want to add more keywords to the set
%  numbers=left,                    % where to put the line-numbers; possible values are (none, left, right)
  numbers=none,
  numbersep=5pt,                   % how far the line-numbers are from the code
  numberstyle=\tiny\color{mygray}, % the style that is used for the line-numbers
  rulecolor=\color{black},         % if not set, the frame-color may be changed on line-breaks within not-black text (e.g. comments (green here))
  showspaces=false,                % show spaces everywhere adding particular underscores; it overrides 'showstringspaces'
  showstringspaces=false,          % underline spaces within strings only
  showtabs=false,                  % show tabs within strings adding particular underscores
  stepnumber=1,                    % the step between two line-numbers. If it's 1, each line will be numbered
  stringstyle=\color{mymauve},     % string literal style
  tabsize=4,                       % sets default tabsize to 4 spaces
  caption=\lstname                   % show the filename of files included with \lstinputlisting; also try caption instead of title
}


\title{1.723 HW3}
\author{Sachith  Dunatunga}

\begin{document}
\maketitle

\section{Problem 1}
\subsection{Part 1}
The equation we want to nondimensionalize is given by
\begin{align}
    c_t \frac{\partial p}{\partial t} + \frac{\partial}{\partial x} \left( \frac{-k}{\mu} \frac{\partial p}{\partial x} \right) = 0.
    \label{eqn:pressure}
\end{align}

We note first that all the variables can be written as $x_D = x / x_c$, $p_D = p / p_c$, and etc.
Moreover, the derivatives are given by $dx_D = dx / x_c$, $dp_D = dp / p_c$, and etc.

Using the chain rule, we can then rewrite equation \eqref{eqn:pressure} as
\begin{align}
    \frac{c_t p_c}{T_c} \frac{\partial p_D}{\partial t_D} + \frac{1}{x_c} \frac{\partial}{\partial x_D} \left( \frac{-k_D k_c}{\mu_D \mu_c} \frac{p_c}{x_c} \frac{\partial p_D}{\partial x_D} \right) = 0
\end{align}
which can be rearranged into
\begin{align}
    \frac{c_t p_c}{T_c} \frac{\partial p_D}{\partial t_D} + \frac{k_c p_c}{x_c^2 \mu_c} \frac{\partial}{\partial x_D} \left( \frac{-k_D}{\mu_D}\frac{\partial p_D}{\partial x_D} \right) = 0.
\end{align}
The coefficients can be collected together into
\begin{align}
    \frac{\partial p_D}{\partial t_D} + \frac{T_c k_c}{c_t x_c^2 \mu_c} \frac{\partial}{\partial x_D} \left( \frac{-k_D}{\mu_D}\frac{\partial p_D}{\partial x_D} \right) = 0.
\end{align}

Setting the term
\begin{align}
    \frac{T_c k_c}{c_t x_c^2 \mu_c}
\end{align}
to unity yields the expression
\begin{align}
    T_c = \frac{c_t x_c^2 \mu_c}{k_c} = \frac{c_t L^2 \mu_c}{k_c}
\end{align}

\subsection{Part 2}
The Darcy velocity is given by
\begin{align}
\mathbf{u} = \frac{-k}{\mu} \nabla p.
\end{align}
In 1D we have
\begin{align}
u = \frac{-k}{\mu} \frac{\partial p}{\partial x}.
\end{align}
Following the same procedure as the time, we arrive at
\begin{align}
u_c u_D = \frac{-k_c k_D}{\mu_c \mu_D} \frac{p_c}{x_c} \frac{\partial p_D}{\partial x_D},
\end{align}
and after collecting terms and substituting we have
\begin{align}
u_c = \frac{k_c p_c}{\mu_c L}.
\end{align}

\subsection{Part 3}
We calculate the above quantities given
\begin{align}
    L &= \num{1e3}\ \si{\meter} \\
    k_c &= \num{1e-13}\ \si{\meter\squared} \\
    \mu_c &= \num{1e-3}\ \si[inter-unit-product = \ensuremath{{}\cdot{}}]{\pascal\second} \\
    c_t &= \num{1e-8}\ \si{\per\pascal} \\
    \Delta P_c &= \num{1e5}\ \si{\pascal}.
\end{align}

Plugging in yields
\begin{align}
    T_c = \frac{c_t L^2 \mu_c}{k_c} = \frac{10^{-8} (10^3)^2 10^{-3}}{10^{-13}}\ \si{\second} = 10^8\ \si{\second}
\end{align}
(which seems a bit long at approximately 3 years) and
\begin{align}
    u_c = \frac{k_c p_c}{\mu_c L} = \frac{10^{-13} 10^5}{10^{-3} 10^3} = 10^{-8}\ \si{\meter\per\second}.
\end{align}


\section{Problem 2}
\subsection{Part 1}

We want to solve the nondimensionalized equation
\begin{align}
    \frac{\partial p}{\partial t} + \frac{\partial}{\partial x} \left( -\lambda \frac{\partial p}{\partial x} \right) = 0
\end{align}
(with $\lambda = 1$) subject to the boundary conditions
\begin{align}
    p(0, t) &= 1 \\
    -\lambda \frac{\partial p}{\partial x}\bigg\rvert_{x=1} &= 0
\end{align}
for $t > 0$ and initial condition
\begin{align}
p(x, 0) = 0.
\end{align}

We note that the boundary conditions are not homogenous.
We first calculate the steady state solution $p_s(x)$ given by
\begin{align}
    \frac{\partial^2 p_s}{\partial x^2} = 0
\end{align}
with boundary conditions
\begin{align}
    p_s(0) &= 1 \\
    -\frac{\partial p_s}{\partial x}\bigg\rvert_{x=1} &= 0.
\end{align}
The general solution is found by integrating twice to obtain
\begin{align}
    p_s(x) = Ax + B.
\end{align}
Using the BCs we obtain $p_s(0) = B = 1$ and $p_s'(1) = A = 0$, so the steady state solution is the constant $p_s(x) = 1$.

We now transform the equation for $p(x,t)$ to an equation for $p_t(x,t)$ given by $p(x,t) = p_t(x,t) + p_s(x)$.
Plugging in and substituting yields
\begin{align}
    \frac{\partial p_t}{\partial t} + \frac{\partial}{\partial x} \left( - \frac{\partial p_t}{\partial x} \right) = 0
    \label{eqn:p-transient}
\end{align}
with boundary conditions
\begin{align}
    p_t(0, t) &= 0 \\
    -\frac{\partial p_t}{\partial x}\bigg\rvert_{x=1} &= 0
\end{align}
(which are homogenous) and initial condition
\begin{align}
    p_t(x, 0) = -1.
\end{align}

We solve equation \eqref{eqn:p-transient} by using separation of variables. Assume we can decompose as $p_t(x,t) = X(x)T(t)$.
We then arrive at the equation
\begin{align}
    X(x)T'(t) - T(t)X''(x) = 0,
\end{align}
which can be expressed as the pair of equations
\begin{align}
    \frac{X''(x)}{X(x)} = \frac{T'(t)}{T(t)} = -\omega^2
\end{align}
We can use the characteristic roots to obtain the general solutions $X(x) = A\sin(\omega x) + B\cos(\omega x)$ and $T(t) = C\exp(-\omega^2 t)$, where $A$, $B$, and $C$ are arbitrary constants.
Without loss of generality we can absorb the constant $C$ to obtain the general solution for $p_t(x,t) = (\tilde{A}\sin(\omega x) + \tilde{B}\cos(\omega x))\exp(-\omega^2 t)$.
This solution must satisfy the boundary conditions, so we know that
\begin{align}
    p_t(0, t) &= 0 \\
    \tilde{B} \exp(-\omega^2 t) &= 0 \implies \tilde{B} = 0
\end{align}
and
\begin{align}
    -\frac{\partial p_t}{\partial x}\bigg\rvert_{x=1} &= 0 \\
    \tilde{A}\omega \cos(\omega) &= 0.
\end{align}
Since we don't want the trivial solution (when $\tilde{A} = 0$), we instead set $\cos(\omega) = 0$ by forcing special values of $\omega$.
In this case, we want $\omega = \frac{\pi}{2} (2n - 1)$ where $n$ is a whole number.
This generates a new solution for each $\omega$, which each satisfy the (homogenous) PDE, so we can sum them together and still get a solution.
Thus, we can write
\begin{align}
    p_t(x,t) = \sum_{n=1}^{\infty} a_n \sin\left(\frac{\pi}{2} (2n - 1) x\right) \exp\left(-\frac{\pi^2}{4} (2n - 1)^2 t \right)
\end{align}

We determine the $a_n$ by first noting that
\begin{align}
    \int_0^1 \sin\left(\frac{\pi}{2} (2n - 1) x\right) \sin\left(\frac{\pi}{2} (2m - 1) x\right) \mathrm{dx} = \frac{1}{2} \delta_{mn}
\end{align}
where $\delta_{mn} = 1$ if $m = n$ and $0$ otherwise.
The orthogonality of the sines allows us to determine $a_n$ by simply integrating the initial conditions against the corresponding sine function, yielding
\begin{align}
a_n &= 2 \int_0^1 (-1) \sin\left(\frac{\pi}{2} (2n - 1) x\right) \mathrm{dx} \\
    &= -2 \frac{2}{\pi (2n - 1)} \cos\left(\frac{\pi}{2} (2n - 1) x\right) \bigg\rvert_{x=0}^{x=1} \\
a_n &= \frac{-4}{\pi (2n - 1)}.
\end{align}

The complete solution for $p_t(x,t)$ is then
\begin{align}
    p_t(x,t) = \sum_{n=1}^{\infty} \frac{-4}{\pi (2n - 1)} \sin\left(\frac{\pi}{2} (2n - 1) x\right) \exp\left(-\frac{\pi^2}{4} (2n - 1)^2 t \right)
\end{align}
and likewise $p(x,t) = p_t(x,t) + p_s(x)$ is given by
\begin{align}
    p(x,t) = 1 + \sum_{n=1}^{\infty} \frac{-4}{\pi (2n - 1)} \sin\left(\frac{\pi}{2} (2n - 1) x\right) \exp\left(-\frac{\pi^2}{4} (2n - 1)^2 t \right)
\end{align}

When numerically evaulating this, we note that increasing $n$ results in a smaller contribution to the total solution, since each term has at least an $\exp\left(-\frac{\pi^2}{4} (2n - 1)^2 t \right)$ factor and sine is bounded to have a magnitude of 1 at most.
Moreover, the $a_n$ have a maximum magnitude of $-4/\pi$ at $n=1$.
This means we can conservatively estimate the maximum contribution of term $n$ as the exponential part alone.
If we are given a time $t$, we can ensure that no term beyond $n$ contributes more than $\epsilon$ error when we reach term
\begin{align}
n = \Bigg\lceil \frac{1}{2}\left(1 + \frac{2}{\pi} \sqrt{\frac{-\log{\epsilon}}{t}}\right) \Bigg\rceil
\end{align}
and moreover this converges quickly since each term is exponentially smaller than the previous one.
We implemented this in the plotting code used to generate figure \ref{fig:p_analytic}

\begin{figure}[!ht]
    \centering
    \includegraphics[scale=1.0]{p_over_time.pdf}
    \caption{The nondimensional pressure distribution $p(x,t)$ is plotted at various nondimensional times.
    We see that even by $t=2.0$, the solution is almost completely at the steady-state solution of $p(x,t) = 1$.}
    \label{fig:p_analytic}
\end{figure}

% \appendix
% \section{Programs}
% \lstinputlisting[label=code:plotting]{sandstone_figures.py}

\end{document}
